% Copyright (c) 2025 Quan-feng WU <wuquanfeng@ihep.ac.cn>
% 
% This software is released under the MIT License.
% https://opensource.org/licenses/MIT

\documentclass{article}

\usepackage{amsmath}
% \usepackage{amsfonts}
\usepackage{amssymb}
\usepackage{authblk}
\usepackage[sorting=none, style=numeric-comp]{biblatex}
    \addbibresource{from_inspirehep.bib}
    \addbibresource{others.bib}
\usepackage[inline]{enumitem}
\usepackage{fontawesome5}
\usepackage[a4paper, margin=1.5cm]{geometry}
\usepackage{graphicx}
\usepackage[
    colorlinks=true,
    anchorcolor=violet,
    citecolor=orange,
    linkcolor=blue,
    urlcolor=magenta
]{hyperref}
% \usepackage[color={1 0 .5}]{attachfile2}
% \usepackage{mathrsfs}
\usepackage{mathtools}
\usepackage{physics}
\usepackage{slashed}
\usepackage{subcaption}
\usepackage{tensor}
\usepackage{xcolor}

\newcommand{\email}[1]{\footnote{E-mail: \href{mailto:#1}{#1}}}
\newcommand{\eg}{\textit{e.g.}}
\newcommand{\etc}{\textit{etc.}}
\newcommand{\ie}{\textit{i.e.}}
\newcommand{\license}{\footnote{This note © 2025 by \href{https://github.com/Fenyutanchan}{Quan-feng WU} is licensed under \href{https://creativecommons.org/licenses/by-nc-/4.0/}{CC BY-NC-ND 4.0}. \faCreativeCommons \faCreativeCommonsBy \faCreativeCommonsNc \faCreativeCommonsNd}}

\DeclareDocumentCommand{\TBA}{o}{{\color{red} TBA\IfNoValueTF{#1}{.}{: #1}}}
\DeclareDocumentCommand{\TBD}{o}{{\color{red} TBD\IfNoValueTF{#1}{.}{: #1}}}

\newcommand{\wqf}[1]{{\color{teal} [WQF: #1]}}
\newcommand{\notsure}[1]{{\color{green!60!black!100!} #1}}

\newcommand{\ee}{\mathrm{e}}
\newcommand{\ii}{\mathrm{i}}

\DeclareDocumentCommand{\twopi}{o}{\IfNoValueTF{#1}{2 \pi}{\qty(2 \pi)^{#1}}}
\DeclareDocumentCommand{\diracdelta}{om}{\delta\IfNoValueF{#1}{^{(#1)}}\qty(#2)}
\DeclareDocumentCommand{\twopidelta}{om}{\IfNoValueTF{#1}{\twopi[] \diracdelta{#2}}{\twopi[#1] \diracdelta[#1]{#2}}}
\DeclareDocumentCommand{\Heaviside}{sm}{\IfBooleanTF{#1}{\widetilde{\Theta}}{\Theta}\qty(#2)}

\newcommand{\GF}{G_\mathrm{F}}
\newcommand{\GN}{G_\mathrm{N}}
\newcommand{\MPl}{M_\mathrm{Pl}}
\newcommand{\kB}{k_\mathrm{B}}
\newcommand{\mPl}{m_\mathrm{Pl}}
\newcommand{\tPl}{t_\mathrm{Pl}}

\newcommand{\eV}{\mathrm{eV}}
\newcommand{\gram}{\mathrm{g}}
\newcommand{\meter}{\mathrm{m}}
\DeclareDocumentCommand{\second}{s}{\mathrm{s\IfBooleanT{#1}{ec}}}

\newcommand{\centi}{\mathrm{c}}
\newcommand{\kilo}{\mathrm{k}}
\newcommand{\mega}{\mathrm{M}}
\newcommand{\giga}{\mathrm{G}}
\newcommand{\tera}{\mathrm{T}}

\newcommand{\keV}{\kilo\eV}
\newcommand{\MeV}{\mega\eV}
\newcommand{\GeV}{\giga\eV}
\newcommand{\TeV}{\tera\eV}
\newcommand{\cm}{\centi\meter}
\newcommand{\kg}{\kilo\gram}

\newcommand{\tento}[1]{10^{#1}}
\newcommand{\timestento}[1]{\times \tento{#1}}

\newcommand{\hc}{\mathrm{h.c.}}

\newcommand{\tildedp}[1]{\widetilde{\dd{#1}}}

\newcommand{\BG}{\mathrm{BG}}
\newcommand{\BH}{\mathrm{BH}}
\newcommand{\PBH}{\mathrm{PBH}}
\newcommand{\QCD}{\mathrm{QCD}}
\newcommand{\tot}{\mathrm{tot}}

\newcommand{\element}[2]{\tensor[^{#2}]{\mathrm{#1}}{}}

\newcommand{\calC}{\mathcal{C}}
\newcommand{\calM}{\mathcal{M}}
\newcommand{\frakf}{\mathfrak{f}}
\newcommand{\rmk}{\mathrm{k}}
\newcommand{\rmp}{\mathrm{p}}
\newcommand{\rmq}{\mathrm{q}}

\newcommand{\amp}[1]{\calM_{#1}}
\newcommand{\ampNormSqr}[1]{\abs{\amp{#1}}^2}
\newcommand{\sumAmpNormSqr}[1]{\sum\ampNormSqr{#1}}
\newcommand{\avgAmpNormSqr}[1]{\overline{\ampNormSqr{#1}}}

\DeclareDocumentCommand{\CT}{oooo}{\calC\IfNoValueF{#2}{_{#2}}\IfNoValueF{#4}{^{#4}}\IfNoValueF{#1}{\qty[#1]}\IfNoValueF{#3}{\qty(#3)}}
\DeclareDocumentCommand{\Lagrangian}{oo}{\mathcal{L}\IfNoValueF{#1}{_{#1}}\IfNoValueF{#2}{\qty[#2]}}

\newcommand{\GW}{\mathrm{GW}}

\newcommand{\eff}{\mathrm{eff}}
\newcommand{\ini}{\mathrm{ini}}
\newcommand{\fin}{\mathrm{fin}}

\newcommand{\upini}{{(\ini)}}


\title{Note on Collision Term with Bose-Einstein Enhancement Phase Space Distribution}
\author{Quan-feng WU\email{wuquanfeng@ihep.ac.cn}}
\affil{
    Institute of High Energy Physics, Chinese Academy of Sciences, \\
    Beijing 100049, China
}
\date{\today\license}

\begin{document}

\maketitle

\begin{abstract}
    \TBA
\end{abstract}
\noindent\hrulefill

\tableofcontents
\noindent\hrulefill
\clearpage

\section{Preliminaries}

We are now consider the Boltzmann equation for gravitons as
\begin{equation}
    \pdv{f_h}{t} - H \rmp_h \pdv{f_H}{\rmp_h} = \CT[f_h].
    \label{eq:BTE-for-f_h}
\end{equation}
where $\CT[f_h]$ is the corresponding collision term and $H$ is the Hubble parameter.

The phase space distribution of the reheaton $\varphi$ is given by
\begin{equation}
    f_\varphi(t, \rmp_\varphi) = \tilde{f}_\varphi(a, \rmp_\varphi) = \exp[\frac{\pi \lambda^2 n_\phi(a')}{m_\phi^4 H(a')} \tilde{\Theta}\qty(\frac{m_\phi}{2} \frac{a_\ini}{a} \le \rmp_\varphi \le \frac{m_\phi}{2})]_{a' = 2 \rmp_\varphi a / m_\phi} - 1,
    \label{eq:f_reheaton}
\end{equation}
where $n_\phi(a)$ is the number density of the inflaton that obeys
\begin{equation}
    \dv{a} \qty(n_\phi a^3) = -\frac{\lambda^2}{32 \pi m_\phi a H(a)} \exp[\frac{\pi \lambda^2}{m_\phi^4 a^3 H(a)} \qty(n_\phi a^3)] \qty(n_\phi a^3).
\end{equation}
According to the exponential factor, the value of $n_\phi a^3 / a_\ini^3$ before inflaton decay is evaluated by
\begin{equation}
    n_\phi(a) \times \qty(\frac{a}{a_\ini})^3 \simeq \left\{
        \begin{aligned}
            & \frac{3 \MPl^2 H_\ini^2}{m_\phi} & \qfor \lambda \le \lambda_\mathrm{c}, \\
            & \frac{m_\phi^4 H_\ini}{\pi \lambda^2} \ln\frac{32 \pi m_\phi H_\ini}{\lambda^2} & \qfor \lambda \ge \lambda_\mathrm{c}
        \end{aligned}
    \right.
\end{equation}
with
\begin{equation}
    \lambda_\mathrm{c} \equiv \frac{m_\phi^5}{3 \pi \MPl^2 H_\ini} \mathcal{W}\qty[\frac{96 \pi^2 \MPl^2 H_\ini^2}{m_\phi^4}],
\end{equation}
where $\mathcal{W}(x)$ is the Lambert function that satisfies $\mathcal{W}(x) e^{\mathcal{W}(x)} = x$.
For simplicity, the evolution of $n_\phi a^3$ can be approximated by
\begin{equation}
    \begin{aligned}
        n_\phi(a) \times \qty(\frac{a}{a_\ini})^3 & \simeq \min\qty(\frac{3 \MPl^2 H_\ini^2}{m_\phi}, \frac{m_\phi^4 H_\ini}{\pi \lambda^2} \ln\frac{32 \pi m_\phi H_\ini}{\lambda^2}) \exp[-\Gamma_{\phi \to \varphi \varphi} (t - t_\ini)] \\
        & \simeq n_\phi'^{(\ini)} \exp[-\frac{C_H \lambda^2}{32 \pi m_\phi H(a)}],
    \end{aligned}
\end{equation}
where we have applied $t - t_\ini \simeq t$ for $t \gg t_\ini$ and $H = C_H / t$ with $C_H = 2/3$ for the matter-dominated (MD) era or $C_H = 1/2$ during the radiation-dominated (RD) era.
And we have defined that
\begin{equation}
    n_\phi'^{(\ini)} \equiv \min\qty(\frac{3 \MPl^2 H_\ini^2}{m_\phi}, \frac{m_\phi^4 H_\ini}{\pi \lambda^2} \ln\frac{32 \pi m_\phi H_\ini}{\lambda^2}).
\end{equation}
If the evolution history of the Universe is purely MD or RD, the Hubble parameter can be simply written as
\begin{equation}
    H(a) = H_\ini \qty(\frac{a_\ini}{a})^{C_H^{-1}}.
\end{equation}
Therefore,
\begin{equation}
    \begin{aligned}
        f_\varphi(a, \rmp_\varphi) & \simeq \exp{\frac{\pi \lambda^2 n_\phi'^{(\ini)}}{m_\phi^4 H_\ini} \qty(\frac{a_\ini}{a'})^{3 - C_H^{-1}} \exp[-\frac{C_H \lambda^2}{32 \pi m_\phi H_\ini} \qty(\frac{a'}{a_\ini})^{C_H^{-1}}] \tilde{\Theta}\qty(a_\ini \le a' \le a)}_{a' = 2 \rmp_\varphi a / m_\phi} - 1 \\
        & =  \exp{\frac{\pi \lambda^2 n_\phi'^{(\ini)}}{m_\phi^4 H_\ini} \qty(\frac{m_\phi a_\ini}{2 \rmp_\varphi a})^{3 - C_H^{-1}} \exp[-\frac{C_H \lambda^2}{32 \pi m_\phi H_\ini} \qty(\frac{2 \rmp_\varphi a}{m_\phi a_\ini})^{C_H^{-1}}] \tilde{\Theta}\qty(\frac{m_\phi}{2} \frac{a_\ini}{a} \le \rmp_\varphi \le \frac{m_\phi}{2})} - 1.
    \end{aligned}
\end{equation}

\section{\boldmath \texorpdfstring{$\varphi + \varphi \to h + h$}{varphi + varphi -> h + h}}

For the process $\varphi(p_1) + \varphi(p_2) \to h(k_1) + h(k_2)$, the collision term is given by
\begin{equation}
    \begin{aligned}
        \CT[f_h][\varphi \varphi \to h h] & = & & \frac{1}{2 E_1^{(h)}} \int \tildedp{p_1} ~ \tildedp{p_2} ~ \tildedp{k_2} ~ \twopidelta[4]{p_1 + p_2 - k_1 - k_2} \\
        & & & \qquad {} \times \frac{S_\mathrm{R}^{(h)}}{S_\mathrm{L} S_\mathrm{R}} \sumAmpNormSqr{\varphi(p_1) + \varphi(p_2) \to h(k_1) + h(k_2)} f_\varphi(p_1) f_\varphi(p_2) \\
        & = & & \frac{1}{4 E_1^{(h)}} \int \tildedp{p_1} ~ \tildedp{p_2} ~ \tildedp{k_2} ~ \twopidelta[4]{p_1 + p_2 - k_1 - k_2} \\
        & & & \qquad {} \times \sumAmpNormSqr{\varphi(p_1) + \varphi(p_2) \to h(k_1) + h(k_2)} f_\varphi(p_1) f_\varphi(p_2),
    \end{aligned}
\end{equation}
where we have applied $S_\mathrm{L} = 2!$ and $S_\mathrm{R} = S_\mathrm{R}^{(h)} = 2!$ for the identical particles in the left and right sides of the reaction, respectively.
The corresponding norm squared of amplitude that summed over the left- and right-hand side degrees of freedom is given by \cite[Eq.~(A.19)]{Bernal:2025lxp}
\begin{equation}
    \sumAmpNormSqr{\varphi(p_1) + \varphi(p_2) \to h(k_1) + h(k_2)} = \frac{2 t^2 u^2}{\MPl^4 s^2} = \frac{2 t^2 \qty(s + t)^2}{\MPl^4 s^2},
\end{equation}
where we have applied that $s + t + u = 0$ for all the external particles being massless.

First of all, we insert two identities of
\begin{align}
    1 & \equiv \int \frac{\dd[4]{q}}{\twopi[4]} \twopidelta[4]{q - p_1 - p_2}, \\
    1 & \equiv \int \frac{\dd{s}}{\twopi} \twopidelta{s + q^2}
\end{align}
into the integration of $\CT[f_h][\varphi \varphi \to h h]$ to obtain
\begin{equation}
    \begin{aligned}
        & & & \CT[f_h][\varphi \varphi \to h h] \\
        & = & & \frac{1}{4 E_1^{(h)}} \int \frac{\dd{s}}{\twopi} \twopidelta{s + q^2} \int \frac{\dd[4]{q}}{\twopi[3]} \twopidelta[4]{q - p_1 - p_2} \int \tildedp{p_1} ~ \tildedp{p_2} ~ \tildedp{k_2} ~ \twopidelta[4]{q - k_1 - k_2} \\
        & & & \qquad {} \times \sumAmpNormSqr{\varphi(p_1) + \varphi(p_2) \to h(k_1) + h(k_2)} f_\varphi(p_1) f_\varphi(p_2) \\
        & = & & \frac{1}{4 E_1^{(h)}} \int_0^\infty \frac{\dd{s}}{\twopi} \int \tildedp{q} \int \tildedp{p_1} ~ \tildedp{p_2} ~ \twopidelta[4]{q - p_1 - p_2} f_\varphi(p_1) f_\varphi(p_2) \\
        & & & \qquad {} \times \int \tildedp{k_2} ~ \twopidelta[4]{q - k_1 - k_2} \frac{2 t^2 \qty(s + t)^2}{\MPl^4 s^2}. \\
    \end{aligned}
\end{equation}
First, we have
\begin{equation}
    \begin{aligned}
        & & & \frac{1}{4 \rmk_1} \int \frac{\dd{s}}{\twopi} \int \tildedp{q} ~ \tildedp{\rmk_2} \twopidelta[4]{q - k_1 - k_2} \\
        & = & & \frac{1}{4 \rmk_1} \int_0^\infty \frac{\dd{s}}{\twopi} \int \tildedp{q} ~ \eval{\frac{1}{2 \rmk_2} \twopidelta{\sqrt{\rmq^2 + s} - \rmk_1 - \rmk_2}}_{\rmk_2 = \sqrt{\rmq^2 + \rmk_1^2 - 2 \rmq \rmk_1 \cos \expval{\vb{q}, \vb{k}_1}}} \\
        & = & & \frac{1}{4 \rmk_1} \int_0^\infty \frac{\dd{s}}{\twopi} \int_{-1}^{+1} \dd{\cos \expval{\vb{q}, \vb{k}_1}} \int_0^\infty \frac{\rmq^2 \dd{\rmq}}{\twopi[2] 2 \sqrt{\rmq^2 + s}} \eval{\frac{1}{2 \rmk_2} \twopidelta{\sqrt{\rmq^2 + s} - \rmk_1 - \rmk_2}}_{\rmk_2 = \sqrt{\rmq^2 + \rmk_1^2 - 2 \rmq \rmk_1 \cos \expval{\vb{q}, \vb{k}_1}}} \\
        & = & & \frac{1}{4 \rmk_1} \int_0^\infty \frac{\dd{s}}{\twopi} \int_0^\infty \frac{\rmq^2 \dd{\rmq}}{\twopi[2] 2 \sqrt{\rmq^2 + s}} \frac{1}{2 \rmk_2} \frac{\twopi \rmk_2}{\rmq \rmk_1} \tilde{\Theta}\qty(\rmq \ge \frac{\abs{s - 4 \rmk_1^2}}{4 \rmk_1}) \\
        & = & & \frac{1}{16 \twopi[2] \rmk_1^2} \int_0^\infty \dd{s} \int_{\qty(s + 4 \rmk_1^2) / (4 \rmk_1)}^\infty \dd{E_q} \\
        & = & & \frac{1}{16 \twopi[2] \rmk_1^2} \int_{\rmk_1}^\infty \dd{E_q} \int_0^{4 \rmk_1 \qty(E_q - \rmk_1)} \dd{s}.
    \end{aligned}
\end{equation}
Taking it back into the integration of $\CT[f_h][\varphi \varphi \to h h]$, we have
\begin{equation}
    \begin{aligned}
        & & & \CT[f_h][\varphi \varphi \to h h] \\
        & = & & \frac{1}{16 \twopi[2] \rmk_1^2} \int_{\rmk_1}^\infty \dd{E_q} \int_0^{4 \rmk_1 (E_q - \rmk_1)} \dd{s} \\
        & & & \qquad {} \times \int \tildedp{p_1} ~ \tildedp{p_2} ~ \twopidelta[4]{q - p_1 - p_2} f_\varphi(p_1) f_\varphi(p_2) \frac{2 t^2 \qty(s + t)^2}{\MPl^4 s^2}. \\
    \end{aligned}
\end{equation}
Then, we can continue to calculate the integration over $p_1$ and $p_2$ as
\begin{equation}
    \begin{aligned}
        \int \tildedp{p_1} ~ \tildedp{p_2} ~ \twopidelta[4]{q - p_1 - p_2} & = & & \int \frac{\tildedp{p_1}}{2 E_2^{(\varphi)}} \twopidelta{E_q - E_1^{(\varphi)} - E_2^{(\varphi)}} \\
        & = & & \int \frac{\tildedp{p_1}}{2 \rmp_2} \twopidelta{E_q - \rmp_1 - \rmp_2} \\
        & = & & \int_0^\infty \frac{\rmp_1 \dd{\rmp_1}}{\twopi[3] 4 \rmp_2} \int_{-1}^{+1} \cos \expval{\vb{q}, \vb{p}_1} \int_{0}^{2 \pi} \dd{\varphi} \\
        & & & \qquad {} \times \twopidelta{E_q - \rmp_1 - \sqrt{\rmq^2 + \rmp_1^2 - 2 \rmq \rmp_1 \cos \expval{\vb{q}, \vb{p}_1}}} \\
        & = & & \frac{1}{4 \twopi[2]} \int_{0}^{2 \pi} \dd{\varphi} \int_0^\infty \frac{\rmp_1 \dd{\rmp_1}}{\rmp_2} \frac{\rmp_2}{\rmq \rmp_1} \tilde{\Theta}\qty(\abs{\rmp_1 - \frac{E_q}{2}} \le \frac{\rmq}{2}) \\
        & = & & \frac{1}{4 \twopi[2] \rmq} \int_{0}^{2 \pi} \dd{\varphi} \int_{(E_q - \rmq) / 2}^{(E_q + \rmq) / 2} \dd{\rmp_1},
    \end{aligned}
\end{equation}
where the azimuthal angle $\varphi$ is not integrated out since the integrand depends on it, and we will return to it later.
Therefore,
\begin{equation}
    \begin{aligned}
        & & & \CT[f_h][\varphi \varphi \to h h] \\
        & = & & \frac{1}{64 \twopi[4] \rmk_1^2} \int_{\rmk_1}^\infty \dd{E_q} \int_0^{4 \rmk_1 (E_q - \rmk_1)} \frac{\dd{s}}{\rmq} \int_{(E_q - \rmq) / 2}^{(E_q + \rmq) / 2} \dd{\rmp_1} f_\varphi(\rmp_1) f_\varphi\qty(E_q - \rmp_1) \int_{0}^{2 \pi} \dd{\varphi} \frac{2 t^2 \qty(s + t)^2}{\MPl^4 s^2} \\
        & = & & \frac{1}{64 \twopi[4] \rmk_1^2} \int_{\rmk_1}^\infty \dd{E_q} \int_0^{E_q} \dd{\rmp_1} f_\varphi(\rmp_1) f_\varphi\qty(E_q - \rmp_1) \int_0^{s_{\max}} \frac{\dd{s}}{\rmq} \int_{0}^{2 \pi} \dd{\varphi} \frac{2 t^2 \qty(s + t)^2}{\MPl^4 s^2},
    \end{aligned}
    \label{eq:CT-f_h-analytical}
\end{equation}
where the Mandelstam variable $t$ is evaluated as
\begin{equation}
    \begin{aligned}
        t \coloneq & -\qty(p_1 - k_1)^2 \\
        = & -2 \rmp_1 \rmk_1 \qty(1 - \cos \expval{\vb{p}_1, \vb{k}_1})
    \end{aligned}
\end{equation}
and
\begin{equation}
    s_{\max} = 4 \min\qty[\rmk_1 (E_q - \rmk_1), \rmp_1 \qty(E_q - \rmp_1)].
\end{equation}
With spherical law of cosines, we have
\begin{equation}
    \begin{aligned}
        \cos \expval{\vb{p}_1, \vb{k}_1} & = \cos \expval{\vb{q}, \vb{p}_1} \cos \expval{\vb{q}, \vb{k}_1} + \sin \expval{\vb{q}, \vb{p}_1} \sin \expval{\vb{q}, \vb{k}_1} \cos \varphi \\
        & = \cos \expval{\vb{q}, \vb{p}_1} \cos \expval{\vb{q}, \vb{k}_1} + \sqrt{\qty(1 - \cos^2 \expval{\vb{q}, \vb{p}_1}) \qty(1 - \cos^2 \expval{\vb{q}, \vb{k}_1})} \cos \varphi,
    \end{aligned}
\end{equation}
where $\varphi$ is the azimuthal angle between the planes of $(\vb{q}, \vb{p}_1)$ and $(\vb{q}, \vb{k}_1)$.
Notice that $\expval{\vb{q}, \vb{k}_1}$ and $\expval{\vb{q}, \vb{p}_1}$ are in the range of $[0, \pi]$, so
\begin{equation*}
    \sin \expval{\vb{q}, \vb{k}_1} = +\sqrt{1 - \cos^2 \expval{\vb{q}, \vb{k}_1}} \qand \sin \expval{\vb{q}, \vb{p}_1} = + \sqrt{1 - \cos^2 \expval{\vb{q}, \vb{p}_1}}
\end{equation*}
are applied.
There are two cosine terms that have been derived before, and we present them here explicitly as
\begin{align}
    \cos \expval{\vb{q}, \vb{k}_1} & = \frac{2 \rmk_1 E_q - s}{2 \rmk_1 \rmq }, \\
    \cos \expval{\vb{q}, \vb{p}_1} & = \frac{2 \rmp_1 E_q - s}{2 \rmp_1 \rmq}.
\end{align}
To evaluate the integration over $\varphi$, we first consider the integration over $t \equiv A + B \cos \varphi$ as
\begin{equation}
    \int_0^{2 \pi} \dd{\varphi} \frac{2 t^2 (s + t)^2}{\MPl^4 s^2} = \frac{\pi}{2 \MPl^4 s^2} \qty[8 A^4 + 3 B^4 + 24 A^2 B^2 + 8 s A (2 A^2 + 3 B^2) + 4 s^2 (2 A^2 + B^2)].
\end{equation}
Finally, we identify that
\begin{align}
    A & \equiv -2 \rmp_1 \rmk_1 \qty(1 - \cos \expval{\vb{q}, \vb{p}_1} \cos \expval{\vb{q}, \vb{k}_1}), \\
    B & \equiv 2 \rmp_1 \rmk_1 \sqrt{\qty(1 - \cos^2 \expval{\vb{q}, \vb{p}_1}) \qty(1 - \cos^2 \expval{\vb{q}, \vb{k}_1})},
\end{align}
and the corresponding integral is obtained.
Notice that the integration over $\dd{s}$ can also be performed analytically, but the result is too lengthy to be presented here.

To solve the Boltzmann equation shown in Eq.~\eqref{eq:BTE-for-f_h}, we can consider the change of variables as
\begin{equation}
    \begin{aligned}
        t & \to a(t), \\
        p_h & \to \tilde{p}_h \equiv p_h a(t),
    \end{aligned}
\end{equation}
The Boltzmann equation now becomes
\begin{equation}
    H a \pdv{\tilde{f}_h}{a}\qty(a, \tilde{p}_h) = \CT[f_h][][a, \tilde{p}_h],
\end{equation}
whose solution is simply
\begin{equation}
    \tilde{f}_h\qty(a, \tilde{p}_h) = \int_{a_\ini}^{a} \frac{\dd{\alpha}}{H(\alpha) \alpha} \CT[f_h][][\alpha, \tilde{p}_h].
\end{equation}
Applying the collision term of Eq.~\eqref{eq:CT-f_h-analytical}, we have
\begin{equation}
    \begin{aligned}
        \tilde{f}_h\qty(a, \tilde{k}_1) & = & & \int_{a_\ini}^{a} \frac{\dd{\alpha}}{H(\alpha) \alpha} \frac{1}{64 \twopi[4] \rmk_1^2} \int_{\rmk_1}^\infty \dd{E_q} \int_0^{E_q} \dd{\rmp_1} f_\varphi(\rmp_1) f_\varphi\qty(E_q - \rmp_1) \\
        & & & \qquad {} \times \int_0^{s_{\max}} \frac{\dd{s}}{\rmq} \int_{0}^{2 \pi} \dd{\varphi} \frac{2 t^2 \qty(s + t)^2}{\MPl^4 s^2} \\
        & = & & \frac{1}{64 \twopi[4]} \int_{a_\ini}^{a} \frac{\alpha \dd{\alpha}}{H(\alpha) \tilde{\rmk}_1^2} \int_{\tilde{\rmk}_1}^\infty \frac{\dd{\tilde{E}_q}}{\alpha} \int_0^{\tilde{E}_q} \frac{\dd{\tilde{\rmp}_1}}{\alpha} \tilde{f}_\varphi(\alpha, \tilde{\rmp}_1) \tilde{f}_\varphi\qty(\alpha, \tilde{E}_q - \tilde{\rmp}_1) \\
        & & & \qquad {} \times \int_0^{\tilde{s}_{\max}} \frac{\dd{\tilde{s}}}{\tilde{\rmq} \alpha} \int_{0}^{2 \pi} \dd{\varphi} \frac{2 \tilde{t}^2 \qty(\tilde{s} + \tilde{t})^2}{\MPl^4 \tilde{s}^2 \alpha^4}
    \end{aligned}
\end{equation}
with
\begin{equation}
    \tilde{s}_{\max} = 4 \min\qty[\tilde{\rmk}_1 (\tilde{E}_q - \tilde{\rmk}_1), \tilde{\rmp}_1 (\tilde{E}_q - \tilde{\rmp}_1)],
\end{equation}
where we have also defined that $\tilde{\rmk}_1 \equiv \rmk_1 a$, $\tilde{E}_q \equiv E_q a$, $\tilde{\rmp}_1 \equiv \rmp_1 a$, $\tilde{s} \equiv s a^2$, $\tilde{\rmq} \equiv \rmq a$, and $\tilde{t} \equiv t a^2$ ($a \to \alpha$ in the integrand).
Here, the phase space distribution of the reheaton $\varphi$ in the comoving momentum is given by
\begin{equation}
    \tilde{f}_\varphi\qty(\alpha, \tilde{\rmp}_\varphi) = \exp[\frac{\pi \lambda^2 n_\phi(\alpha')}{m_\phi^4 H(\alpha')} \tilde{\Theta}\qty(\frac{m_\phi a_\ini}{2} \le \tilde{\rmp}_\varphi \le \frac{m_\phi \alpha}{2})]_{\alpha' = 2 \tilde{\rmp}_\varphi / m_\phi} - 1,
\end{equation}
which requires that
\begin{equation}
    \frac{m_\phi a_\ini}{2} \le \tilde{\rmp}_1 \le \frac{m_\phi \alpha}{2} \qand \frac{m_\phi a_\ini}{2} \le \tilde{E}_q - \tilde{\rmp}_1 \le \frac{m_\phi \alpha}{2}
\end{equation}
for the non-vanishing product of $\tilde{f}_\varphi(\alpha, \tilde{\rmp}_1) \tilde{f}_\varphi(\alpha, \tilde{E}_q - \tilde{\rmp}_1)$.
Summing them up, we have then
\begin{equation}
    m_\phi a_\ini \le \tilde{E}_q \le m_\phi \alpha,
\end{equation}
which also requires $\tilde{\rmk}_1 \le m_\phi \alpha \le m_\phi a$ since $\tilde{\rmk}_1 \le \tilde{E}_q$, \ie,
\begin{equation}
    \max\qty(a_\ini, \frac{\tilde{k}_1}{m_\phi}) \le \alpha \le a.
\end{equation}
Furthermore, the non-vanishing product of $\tilde{f}_\varphi(\alpha, \tilde{\rmp}_1) \tilde{f}_\varphi(\alpha, \tilde{E}_q - \tilde{\rmp}_1)$ also requires
\begin{equation}
    \abs{\tilde{\rmp}_1 - \frac{\tilde{E_q}}{2}} \le \frac{1}{2} \min\qty(m_\phi \alpha - \tilde{E}_q, \tilde{E}_q - m_\phi a_\ini).
\end{equation}
Hence, the integration ranges of $\tilde{\rmk}_1$, $\tilde{\rmq}$ and $\tilde{\rmp}_1$ are constrained, \ie,
\begin{equation}
    \begin{aligned}
        \tilde{f}_h\qty(a, \tilde{\rmk}_1) & = & & \frac{1}{64 \twopi[4] \tilde{\rmk}_1^2} \tilde{\Theta}\qty(0 \le \tilde{\rmk}_1 \le m_\phi a) \int_{\max\qty(a_\ini, \tilde{k}_1 / m_\phi)}^{a} \frac{\dd{\alpha}}{H(\alpha) \alpha^6} \\
        & & & \qquad {} \times \int_{\max\qty(\tilde{\rmk}_1, m_\phi a_\ini)}^{m_\phi \alpha} \dd{\tilde{E}_q} \int_{\tilde{\rmp}_1^{(-)}}^{\tilde{\rmp}_1^{(+)}} \dd{\tilde{\rmp}_1} \tilde{f}_\varphi(\alpha, \tilde{\rmp}_1) \tilde{f}_\varphi\qty(\alpha, \tilde{E}_q - \tilde{\rmp}_1) \\
        & & & \qquad {} \times \int_0^{\tilde{s}_{\max}} \frac{\dd{\tilde{s}}}{\tilde{\rmq}} \int_{0}^{2 \pi} \dd{\varphi} \frac{2 \tilde{t}^2 \qty(\tilde{s} + \tilde{t})^2}{\MPl^4 \tilde{s}^2},
    \end{aligned}
\end{equation}
where
\begin{equation}
    \tilde{\rmp}_1^{(\pm)} = \frac{1}{2} \qty[\tilde{E}_q \pm \min\qty(m_\phi \alpha - \tilde{E}_q, \tilde{E}_q - m_\phi a_\ini)].
\end{equation}

\paragraph{Dirac-$\delta$ Approximation}
Let us consider the approximation that
\begin{equation}
    f_\varphi\qty(t, \rmp_\varphi) \simeq \frac{2 \pi^2}{\rmp_\varphi^2} n_\varphi(t) \delta\qty(\rmp_\varphi - \frac{m_\phi}{2} \frac{a_\ini}{a(t)}),
\end{equation}
or equivalently,
\begin{equation}
    \tilde{f}_\varphi\qty(a, \tilde{\rmp}_\varphi) \simeq \frac{2 \pi^2}{\tilde{\rmp}_\varphi^2} \qty[n_\varphi(a) a^3] \delta\qty(\tilde{\rmp}_\varphi - \frac{m_\phi a_\ini}{2}).
\end{equation}
Applying this approximation, we have
\begin{equation}
    \begin{aligned}
        \tilde{f}_h\qty(a, \tilde{\rmk}_1) & = & & \frac{1}{64 \twopi[4] \tilde{\rmk}_1^2} \tilde{\Theta}\qty(0 \le \tilde{\rmk}_1 \le m_\phi a) \int_{\max\qty(a_\ini, \tilde{k}_1 / m_\phi)}^{a} \frac{\dd{\alpha}}{H(\alpha) \alpha^6} \int_{\max\qty(\tilde{\rmk}_1, m_\phi a_\ini)}^{m_\phi \alpha} \dd{\tilde{E}_q} \\
        & & & \qquad {} \times \int_{\tilde{\rmp}_1^{(-)}}^{\tilde{\rmp}_1^{(+)}} \dd{\tilde{\rmp}_1} \frac{4 \pi^4}{\tilde{\rmp}_1^2 \qty(\tilde{E}_q - \tilde{\rmp}_1)^2} \qty[n_\varphi(\alpha) \alpha^3]^2 \delta\qty(\tilde{\rmp}_1 - \frac{m_\phi a_\ini}{2}) \delta\qty(\tilde{E}_q - \tilde{\rmp}_1 - \frac{m_\phi a_\ini}{2}) \\
        & & & \qquad {} \times \int_0^{\tilde{s}_{\max}} \frac{\dd{\tilde{s}}}{\tilde{\rmq}} \int_{0}^{2 \pi} \dd{\varphi} \frac{2 \tilde{t}^2 \qty(\tilde{s} + \tilde{t})^2}{\MPl^4 \tilde{s}^2} \\
        & = & & \frac{1}{256 \tilde{\rmk}_1^2} \tilde{\Theta}\qty(0 \le \tilde{\rmk}_1 \le m_\phi a) \int_{\max\qty(a_\ini, \tilde{k}_1 / m_\phi)}^{a} \frac{\qty[n_\varphi(\alpha) \alpha^3]^2}{H(\alpha) \alpha^6} \dd{\alpha} \\
        & & & \qquad {} \times \int_{\max\qty(\tilde{\rmk}_1, m_\phi a_\ini)}^{m_\phi \alpha} \dd{\tilde{E}_q} \tilde{\Theta}\qty(\tilde{\rmp}_1^{(-)} \le \frac{m_\phi a_\ini}{2} \le \tilde{\rmp}_1^{(+)}) \qty(\frac{2}{m_\phi a_\ini})^4 \delta\qty(\tilde{E}_q - m_\phi a_\ini) \\
        & & & \qquad {} \times \int_0^{\tilde{s}_{\max}} \frac{\dd{\tilde{s}}}{\tilde{\rmq}} \int_{0}^{2 \pi} \dd{\varphi} \frac{2 \tilde{t}^2 \qty(\tilde{s} + \tilde{t})^2}{\MPl^4 \tilde{s}^2} \\
        & = & & \frac{1}{16 m_\phi^4 a_\ini^4 \tilde{\rmk}_1^2} \tilde{\Theta}\qty(0 \le \tilde{\rmk}_1 \le m_\phi a_\ini) \int_{a_\ini}^{a} \frac{\qty[n_\varphi(\alpha) \alpha^3]^2}{H(\alpha) \alpha^6} \dd{\alpha} \\
        & & & \qquad {} \times \int_0^{4 \tilde{\rmk}_1 \qty(m_\phi a_\ini - \tilde{\rmk}_1)} \frac{\dd{\tilde{s}}}{\tilde{\rmq}} \int_{0}^{2 \pi} \dd{\varphi} \frac{2 \tilde{t}^2 \qty(\tilde{s} + \tilde{t})^2}{\MPl^4 \tilde{s}^2} \\
        & = & & \frac{1}{16 m_\phi^4 a_\ini^4 \tilde{\rmk}_1^2} \tilde{\Theta}\qty(0 \le \tilde{\rmk}_1 \le m_\phi a_\ini) \int_{a_\ini}^{a} \frac{\qty[n_\varphi(\alpha) \alpha^3]^2}{H(\alpha) \alpha^6} \dd{\alpha} \\
        & & & \qquad {} \times \frac{4 \pi}{15 \MPl^4} \qty[
            \begin{aligned}
                m_\phi a_\ini \qty(
                    \begin{aligned}
                        30 \tilde{\rmk}_1^4 - 60 \tilde{\rmk}_1^3 \qty(m_\phi a_\ini) + 40 \tilde{\rmk}_1^2 \qty(m_\phi a_\ini)^2 \\
                        {} - 10 \tilde{\rmk}_1 \qty(m_\phi a_\ini)^3 + \qty(m_\phi a_\ini)^4
                    \end{aligned}
                ) \\
                {} - \abs{m_\phi a_\ini - 2 \tilde{\rmk}_1}^5
            \end{aligned}
        ].
    \end{aligned}
\end{equation}

\section{\boldmath \texorpdfstring{$\varphi + \phi \to \varphi + h$}{varphi + phi -> varphi + h}}

In this section, we consider the process of $\varphi(p_\varphi) + \phi(p_\phi) \to \varphi(k_\varphi) + h(k_h)$, whose squared amplitude is given by\footnote{The expression is not checked yet.}
\begin{equation}
    \sumAmpNormSqr{\varphi + \phi \to \varphi + h} = \frac{\lambda^{2}}{2 \MPl^2} \qty(2 - \frac{m_\phi}{E_h})^2.
\end{equation}
The collision term reads
\begin{equation}
    \begin{aligned}
        & & & \CT[f_h][\varphi \phi \to \varphi h][t, \rmk_h] \\
        & = & & \frac{1}{2 E_h} \int \tildedp{p_\varphi} ~ \tildedp{p_\phi} ~ \tildedp{k_\varphi} ~ \twopidelta[4]{p_\varphi + p_\phi - k_\varphi - k_h} \sumAmpNormSqr{\varphi + \phi \to \varphi + h} f_\varphi(p_\varphi) f_\phi(p_\phi) \qty[1 + f_\varphi(k_\varphi)] \\
        & = & & \frac{1}{2 \rmk_h} \sumAmpNormSqr{\varphi + \phi \to \varphi + h} \int \tildedp{p_\varphi} ~ \tildedp{p_\phi} ~ \tildedp{k_\varphi} ~ \twopidelta[4]{p_\varphi + p_\phi - k_\varphi - k_h} f_\varphi(p_\varphi) f_\phi(p_\phi) \qty[1 + f_\varphi(k_\varphi)].
    \end{aligned}
\end{equation}
Also inserting two identities of
\begin{align}
    1 & \equiv \int \frac{\dd[4]{q}}{\twopi[4]} \twopidelta[4]{q - p_\varphi - p_\phi}, \\
    1 & \equiv \int \frac{\dd{s}}{\twopi} \twopidelta{s + q^2},
\end{align}
we have
\begin{equation}
    \begin{aligned}
        \CT[f_h][\varphi \phi \to \varphi h] & = & & \frac{1}{2 \rmk_h} \sumAmpNormSqr{\varphi + \phi \to \varphi + h} \int \frac{\dd{s}}{\twopi} \int \frac{\dd[4]{q}}{\twopi[4]} \twopidelta{s + q^2} \\
        & & & \qquad {} \times \int \tildedp{p_\varphi} ~ \tildedp{p_\phi} ~ \twopidelta[4]{q - p_\varphi - p_\phi} f_\varphi(p_\varphi) f_\phi(p_\phi) \\
        & & & \qquad {} \times \int \tildedp{k_\varphi} ~ \twopidelta[4]{q - k_\varphi - k_h} \qty[1 + f_\varphi(k_\varphi)] \\
        & = & & \frac{1}{2 \rmk_h} \sumAmpNormSqr{\varphi + \phi \to \varphi + h} \int_0^\infty \frac{\dd{s}}{\twopi} \int \tildedp{q} ~ \tildedp{k_\varphi} ~ \twopidelta[4]{q - k_\varphi - k_h} \qty[1 + f_\varphi(k_\varphi)] \\
        & & & \qquad {} \times \int \tildedp{p_\varphi} ~ \tildedp{p_\phi} ~ \twopidelta[4]{q - p_\varphi - p_\phi} f_\varphi(p_\varphi) f_\phi(p_\phi).
    \end{aligned}
\end{equation}
Here, the integration over $q$ and $k_\varphi$ can be performed as (details are shown in the previous section)
\begin{equation}
    \begin{aligned}
        & & & \int_0^\infty \frac{\dd{s}}{\twopi} \int \tildedp{q} ~ \tildedp{k_\varphi} ~ \twopidelta[4]{q - k_\varphi - k_h} \qty[1 + f_\varphi(k_\varphi)] \\
        & = & & \frac{1}{4 \twopi[2] \rmk_h} \int_0^\infty \dd{s} \int_{\qty(s + 4 \rmk_h^2) / \qty(4 \rmk_h)}^\infty \dd{E_q} \qty[1 + f_\varphi\qty(E_q - \rmk_h)] \\
        & = & & \frac{1}{4 \twopi[2] \rmk_h} \int_{\rmk_h}^\infty \dd{E_q} \int_0^{4 \rmk_h (E_q - \rmk_h)} \dd{s} \qty[1 + f_\varphi\qty(E_q - \rmk_h)].
    \end{aligned}
\end{equation}
Differently, the integration over $p_\varphi$ and $p_\phi$ is not the same as before since $\phi$ is massive, which reads
\begin{equation}
    \begin{aligned}
        & & & \int \tildedp{p_\varphi} ~ \tildedp{p_\phi} ~ \twopidelta[4]{q - p_\varphi - p_\phi} \\
        & = & & \int \eval{\frac{\tildedp{p_\phi}}{2 \rmp_\varphi} \twopidelta{E_q - \rmp_\varphi - \sqrt{\rmp_\phi^2 + m_\phi^2}}}_{\rmp_\varphi = \sqrt{\rmq^2 + \rmp_\phi^2 - 2 \rmq \rmp_\phi \cos \expval{\vb{q}, \vb{p}_\phi}}} \\
        & = & & \int_{-1}^{+1} \dd{\cos \expval{\vb{q}, \vb{p}_\phi}} \int_0^\infty \eval{\frac{\rmp_\phi^2 \dd{\rmp_\phi}}{\twopi[2] 2 E_\phi 2 \rmp_\varphi} \twopidelta{E_q - \rmp_\varphi - \sqrt{\rmp_\phi^2 + m_\phi^2}}}_{\rmp_\varphi = \sqrt{\rmq^2 + \rmp_\phi^2 - 2 \rmq \rmp_\phi \cos \expval{\vb{q}, \vb{p}_\phi}}} \\
        & = & & \int_0^\infty \frac{\rmp_\phi^2 \dd{\rmp_\phi}}{\twopi[] 2 E_\phi 2 \rmp_\varphi} \frac{\rmp_\varphi}{\rmq \rmp_\phi} \tilde{\Theta}\qty(\frac{\abs{\qty(E_q - \rmq)^2 - m_\phi^2}}{2 \qty(E_q - \rmq)} \le \rmp_\phi \le \frac{\qty(E_q + \rmq)^2 - m_\phi^2}{2 \qty(E_q + \rmq)}) \\
        & = & & \frac{1}{4 \twopi[] \rmq} \int_{E_\phi^{(-)}}^{E_\phi^{(+)}} \dd{E_\phi}
    \end{aligned}
\end{equation}
with
\begin{equation}
    E_\phi^{(\pm)} = \frac{\qty(E_q \pm \rmq)^2 + m_\phi^2}{2 \qty(E_q \pm \rmq)}.
\end{equation}
Therefore,
\begin{equation}
    \begin{aligned}
        \CT[f_h][\varphi \phi \to \varphi h] & = & & \frac{1}{32 \twopi[3] \rmk_h^2} \sumAmpNormSqr{\varphi + \phi \to \varphi + h} \int_0^\infty \dd{s} \int_{\qty(s + 4 \rmk_h^2) / \qty(4 \rmk_h)}^\infty \frac{\dd{E_q}}{\rmq} \\
        & & & \qquad {} \times \int_{E_\phi^{(-)}}^{E_\phi^{(+)}} \dd{E_\phi} f_\varphi\qty(E_q - E_\phi) f_\phi\qty(\sqrt{E_\phi^2 - m_\phi^2}) \qty[1 + f_\varphi\qty(E_q - \rmk_h)].
    \end{aligned}
\end{equation}

In our consideration, the distribution of $\phi$ is given by
\begin{equation}
    f_\phi(t, \rmp_\phi) = \frac{2 \pi^2}{\rmp_\phi^2} n_\phi(t) \delta(\rmp_\phi - \epsilon),
\end{equation}
where $\epsilon > 0$ is an infinitesimal to regularize the behavior of $f_\phi(\rmp_\phi)$ at $\rmp_\phi \to 0$.
So, the collision term becomes
\begin{equation}
    \begin{aligned}
        & & & & & \CT[f_h][\varphi \phi \to \varphi h][t, \rmk_h] \\
        & & = & & & \frac{1}{32 \twopi[3] \rmk_h^2} \sumAmpNormSqr{\varphi + \phi \to \varphi + h} \int_0^\infty \dd{s} \int_{\qty(s + 4 \rmk_h^2) / \qty(4 \rmk_h)}^\infty \frac{\dd{E_q}}{\rmq} \\
        & & & & & \qquad {} \times \int_{E_\phi^{(-)}}^{E_\phi^{(+)}} \dd{E_\phi} f_\varphi\qty(E_q - E_\phi) \frac{2 \pi^2}{E_\phi^2 - m_\phi^2} n_\phi(t) \delta\qty(\sqrt{E_\phi^2 - m_\phi^2} - \epsilon) \qty[1 + f_\varphi\qty(E_q - \rmk_h)] \\
        & & = & & & \frac{n_\phi(t)}{64 \twopi[] \rmk_h^2} \sumAmpNormSqr{\varphi + \phi \to \varphi + h} \int_0^\infty \dd{s} \int_{\qty(s + 4 \rmk_h^2) / \qty(4 \rmk_h)}^\infty \frac{\dd{E_q}}{\rmq} \\
        & & & & & \qquad {} \times f_\varphi\qty(E_q - \sqrt{m_\phi^2 + \epsilon^2}) \qty[1 + f_\varphi\qty(E_q - \rmk_h)] \\
        & & & & & \qquad {} \times \frac{\epsilon}{\sqrt{m_\phi^2 + \epsilon^2}} \tilde{\Theta}\qty(\abs{E_q - \epsilon - \sqrt{m_\phi^2 + \epsilon^2}} \le \rmq \le E_q + \epsilon - \sqrt{m_\phi^2 + \epsilon^2}) \frac{1}{\epsilon^2} \\
        & & \tensor[_{\epsilon \to 0}]{=}{} & & & \frac{n_\phi(t)}{64 \twopi[] \rmk_h^2} \sumAmpNormSqr{\varphi + \phi \to \varphi + h} \int_0^\infty \dd{s} \\
        & & & & & \qquad {} \times \int_{\qty(s + 4 \rmk_h^2) / \qty(4 \rmk_h)}^\infty \frac{\dd{E_q}}{\rmq} \frac{s - m_\phi^2}{m_\phi^3} \delta\qty(E_q - \frac{s + m_\phi^2}{2 m_\phi}) f_\varphi(E_q - m_\phi) \qty[1 + f_\varphi\qty(E_q - \rmk_h)] \\
        & & = & & & \frac{n_\phi(t)}{64 \twopi[] \rmk_h^2} \sumAmpNormSqr{\varphi + \phi \to \varphi + h} \int_0^\infty \frac{2 \dd{s}}{m_\phi^2} \tilde{\Theta}\qty(\frac{s + m_\phi^2}{2 m_\phi} \ge \frac{s + 4 \rmk_h^2}{4 \rmk_h}) \\
        & & & & & \qquad {} \times f_\varphi\qty(\frac{s - m_\phi^2}{2 m_\phi}) \qty[1 + f_\varphi\qty(\frac{s + m_\phi^2}{2 m_\phi} - \rmk_h)] \\
        & & = & & & \frac{n_\phi(t)}{32 \twopi[] m_\phi^2 \rmk_h^2} \sumAmpNormSqr{\varphi + \phi \to \varphi + h} \tilde{\Theta}\qty(\rmk_h \ge \frac{m_\phi}{2}) \int_{2 m_\phi \rmk_h}^\infty \dd{s} f_\varphi\qty(\frac{s - m_\phi^2}{2 m_\phi}) \qty[1 + f_\varphi\qty(\frac{s + m_\phi^2}{2 m_\phi} - \rmk_h)].
    \end{aligned}
\end{equation}

\paragraph{\boldmath Dirac-$\delta$ Approximation for \texorpdfstring{$f_\varphi$}{f\_varphi}}
Also considering the Dirac-$\delta$ approximation for $f_\varphi$ as
\begin{equation}
    f_\varphi\qty(t, \rmp_\varphi) \simeq \frac{2 \pi^2}{\rmp_\varphi^2} n_\varphi(t) \delta\qty(\rmp_\varphi - \frac{m_\phi}{2} \frac{a_\ini}{a(t)}),
\end{equation}
the above collision term now becomes
\begin{equation}
    \begin{aligned}
        & & & \CT[f_h][\varphi \phi \to \varphi h][t, \rmk_h] \\
        & \simeq & & \frac{n_\phi(t)}{32 \twopi[] m_\phi^2 \rmk_h^2} \sumAmpNormSqr{\varphi + \phi \to \varphi + h} \tilde{\Theta}\qty(\rmk_h \ge \frac{m_\phi}{2}) \int_{2 m_\phi \rmk_h}^\infty \dd{s} 2 \pi^2 \qty(\frac{2 m_\phi}{s - m_\phi^2})^2 n_\varphi(t) \delta\qty(\frac{s - m_\phi^2}{2 m_\phi} - \frac{m_\phi}{2} \frac{a_\ini}{a(t)}) \\
        & & & \qquad {} \times \qty[1 + \frac{2 \pi^2 n_\varphi(t)}{\qty(\frac{s + m_\phi^2}{2 m_\phi} - \rmk_h)^2} \delta\qty(\frac{s + m_\phi^2}{2 m_\phi} - \rmk_h - \frac{m_\phi}{2} \frac{a_\ini}{a(t)})] \\
        & = & & \frac{\pi n_\phi(t) n_\varphi(t)}{4 m_\phi^3 \rmk_h^2} \qty[\frac{a(t)}{a_\ini}]^2 \sumAmpNormSqr{\varphi + \phi \to \varphi + h} \tilde{\Theta}\qty[\frac{m_\phi}{2} \le \rmk_h \le \frac{m_\phi}{2} \qty(1 + \frac{a_\ini}{a(t)})] \\
        & & & \qquad {} \times \qty(1 + \frac{2 \twopi[2] n_\varphi(t)}{m_\phi^2} \qty[\frac{a(t)}{a_\ini}]^2 \delta\qty(\rmk_h - m_\phi)).
    \end{aligned}
\end{equation}

\paragraph{Direct Integration with Dirac-$\delta$ Approximation}
Now we consider to integrate the collision term directly with the Dirac-$\delta$ approximation for $f_\phi$ and $f_\varphi$ as
\begin{equation}
    \begin{aligned}
        & & & & & \CT[f_h][\varphi \phi \to \varphi h][t, \rmk_h] \\
        & & = & & & \frac{1}{2 E_h} \int \tildedp{p_\varphi} ~ \tildedp{p_\phi} ~ \tildedp{k_\varphi} ~ \twopidelta[4]{p_\varphi + p_\phi - k_\varphi - k_h} \sumAmpNormSqr{\varphi + \phi \to \varphi + h} f_\varphi(p_\varphi) f_\phi(p_\phi) \qty[1 + f_\varphi(k_\varphi)] \\
        & & = & & & \frac{1}{2 \rmk_h} \sumAmpNormSqr{\varphi + \phi \to \varphi + h} \\
        & & & & & \qquad {} \times \int \tildedp{p_\varphi} ~ \tildedp{p_\phi} \eval{\frac{1}{2 \rmk_\varphi} \twopidelta{\rmp_\varphi + E_\phi - \rmk_\varphi - \rmk_h} f_\varphi(p_\varphi) f_\phi(p_\phi) \qty[1 + f_\varphi(\rmk_\varphi)]}_{\rmk_\varphi = \abs{\vb{p}_\varphi + \vb{p}_\phi - \vb{k}_\varphi - \vb{k}_h}} \\
        & & = & & & \frac{1}{2 \rmk_h} \sumAmpNormSqr{\varphi + \phi \to \varphi + h} \int \frac{\dd[3]{\vb{p}_\phi}}{\twopi[3] 2 E_\phi} \frac{2 \pi^2}{\rmp_\phi^2} n_\phi(t) \delta(\rmp_\phi - \epsilon) \int \frac{\dd[3]{\vb{p}_\varphi}}{\twopi[3] 2 \rmp_\varphi} \frac{2 \pi^2}{\rmp_\varphi^2} n_\varphi(t) \delta\qty(\rmp_\varphi - \frac{m_\phi}{2} \frac{a_\ini}{a(t)}) \\
        & & & & & \qquad {} \times \eval{\frac{1}{2 \rmk_\varphi} \twopidelta{\rmp_\varphi + E_\phi - \rmk_\varphi - \rmk_h} \qty[1 + f_\varphi(\rmk_\varphi)]}_{\rmk_\varphi = \abs{\vb{p}_\varphi + \vb{p}_\phi - \vb{k}_\varphi - \vb{k}_h}} \\
        & & = & & & \frac{1}{2 \rmk_h} \sumAmpNormSqr{\varphi + \phi \to \varphi + h} \int_{-1}^{+1} \dd{\cos \expval{\vb{p}_\phi, \vb{k}_h}} \int_{-1}^{+1} \dd{\cos \expval{\vb{p}_\varphi, \vb{k}_h}} \int_0^{2 \pi} \dd{\varphi_{\vb{p}_\phi}} \int_0^{2 \pi} \dd{\varphi_{\vb{p}_\varphi}} \\
        & & & & & \qquad {} \times \int_0^\infty \frac{\rmp_\phi^2 \dd{\rmp_\phi}}{\twopi[3] 2 E_\phi} \frac{2 \pi^2}{\rmp_\phi^2} n_\phi(t) \delta(\rmp_\phi - \epsilon) \int_0^\infty \frac{\rmp_\varphi^2 \dd{\rmp_\varphi}}{\twopi[3] 2 \rmp_\varphi} \frac{2 \pi^2}{\rmp_\varphi^2} n_\varphi(t) \delta\qty(\rmp_\varphi - \frac{m_\phi}{2} \frac{a_\ini}{a(t)}) \\
        & & & & & \qquad {} \times \eval{\frac{1}{2 \rmk_\varphi} \twopidelta{\rmp_\varphi + E_\phi - \rmk_\varphi - \rmk_h} \qty[1 + f_\varphi(\rmk_\varphi)]}_{\rmk_\varphi = \abs{\vb{p}_\varphi + \vb{p}_\phi - \vb{k}_\varphi - \vb{k}_h}} \\
        & & = & & & \frac{1}{2 \rmk_h} \sumAmpNormSqr{\varphi + \phi \to \varphi + h} \int_{-1}^{+1} \dd{\cos \expval{\vb{p}_\phi, \vb{k}_h}} \int_{-1}^{+1} \dd{\cos \expval{\vb{p}_\varphi, \vb{k}_h}} \int_0^{2 \pi} \dd{\varphi_{\vb{p}_\phi}} \int_0^{2 \pi} \dd{\varphi_{\vb{p}_\varphi}} \\
        & & & & & \qquad {} \times \frac{n_\phi(t) n_\varphi(t)}{16 \twopi[2]} \frac{2}{m_\phi \sqrt{m_\phi^2 + \epsilon^2}} \qty[\frac{a(t)}{a_\ini}] \qty[1 + \frac{2 \pi^2}{\rmk_\varphi^2} n_\varphi(t) \delta\qty(\rmk_\varphi - \frac{m_\phi}{2} \frac{a_\ini}{a(t)})] \\
        & & & & & \qquad {} \times\eval{\frac{1}{2 \rmk_\varphi} \twopidelta{\frac{m_\phi}{2} \frac{a_\ini}{a(t)} + \sqrt{\epsilon^2 + m_\phi^2} - \rmk_\varphi - \rmk_h}}_{\rmk_\varphi = \abs{\vb{p}_\varphi + \vb{p}_\phi - \vb{k}_\varphi - \vb{k}_h}} \\
        & & \tensor[_{\epsilon \to 0}]{=}{} & & & \frac{n_\phi(t) n_\varphi(t)}{8 m_\phi^2 \rmk_h} \qty[\frac{a(t)}{a_\ini}] \sumAmpNormSqr{\varphi + \phi \to \varphi + h} \int_{-1}^{+1} \dd{\cos \expval{\vb{p}_\varphi, \vb{k}_h}} \\
        & & & & & \qquad {} \times \eval{\frac{\pi}{\rmk_\varphi} \qty[1 + \frac{2 \pi^2}{\rmk_\varphi^2} n_\varphi(t) \delta\qty(\rmk_\varphi - \frac{m_\phi}{2} \frac{a_\ini}{a(t)})] \delta\qty(\frac{m_\phi}{2} \frac{a_\ini}{a(t)} + m_\phi - \rmk_\varphi - \rmk_h)}_{\rmk_\varphi = \abs{\vb{p}_\varphi - \vb{k}_\varphi - \vb{k}_h}} \\
        & & = & & & \frac{n_\phi(t) n_\varphi(t)}{8 m_\phi^2 \rmk_h} \qty[\frac{a(t)}{a_\ini}] \sumAmpNormSqr{\varphi + \phi \to \varphi + h} \tilde{\Theta}\qty[\frac{m_\phi}{2} \le \rmk_h \le \frac{m_\phi}{2} \qty(1 + \frac{a_\ini}{a(t)})] \\
        & & & & & \qquad {} \times \eval{\frac{\pi}{\rmk_\varphi} \qty[1 + \frac{2 \pi^2}{\rmk_\varphi^2} n_\varphi(t) \delta\qty(\rmk_\varphi - \frac{m_\phi}{2} \frac{a_\ini}{a(t)})] \frac{2 \rmk_\varphi}{m_\phi \rmk_h} \qty[\frac{a(t)}{a_\ini}]}_{\rmk_\varphi = \frac{m_\phi}{2} \frac{a_\ini}{a(t)} + m_\phi - \rmk_h} \\
        & & = & & & \frac{\pi n_\phi(t) n_\varphi(t)}{4 m_\phi^3 \rmk_h^2} \qty[\frac{a(t)}{a_\ini}]^2 \sumAmpNormSqr{\varphi + \phi \to \varphi + h} \tilde{\Theta}\qty[\frac{m_\phi}{2} \le \rmk_h \le \frac{m_\phi}{2} \qty(1 + \frac{a_\ini}{a(t)})] \qty[1 + \frac{2 \pi^2 n_\varphi(t) \delta\qty(m_\phi - \rmk_h)}{\qty(\frac{m_\phi}{2} \frac{a_\ini}{a(t)} + m_\phi - \rmk_h)^2}] \\
        & & = & & & \frac{\pi n_\phi(t) n_\varphi(t)}{4 m_\phi^3 \rmk_h^2} \qty[\frac{a(t)}{a_\ini}]^2 \sumAmpNormSqr{\varphi + \phi \to \varphi + h} \tilde{\Theta}\qty[\frac{m_\phi}{2} \le \rmk_h \le \frac{m_\phi}{2} \qty(1 + \frac{a_\ini}{a(t)})] \\
        & & & & & \qquad {} \times \qty[1 + \frac{8 \pi^2 n_\varphi(t)}{m_\phi^2} \qty[\frac{a(t)}{a_\ini}]^2 \delta\qty(\rmk_h - m_\phi)] .
    \end{aligned}
\end{equation}

\section{\boldmath \texorpdfstring{$\phi \to \varphi + \varphi + h$}{phi -> varphi + varphi + h}}

The squared amplitude of the process $\phi(p_\phi) \to \varphi(k_1) + \varphi(k_2) + h(k_h)$ is given by
\begin{equation}
    \sumAmpNormSqr{\phi \to \varphi + \varphi + h} = \frac{2 \lambda^2}{\MPl^2} \qty(1 - \frac{m_\phi}{2 E_h})^2.
\end{equation}
Then the corresponding collision term reads
\begin{equation}
    \begin{aligned}
        & & & \CT[f_h][\phi \to \varphi \varphi h][t, \rmk_h] \\
        & = & & \frac{1}{2 E_h} \int \tildedp{p_\phi} ~ \tildedp{k_1} ~ \tildedp{k_2} ~ \twopidelta[4]{p_\phi - k_1 - k_2 - k_h} \frac{1}{2} \sumAmpNormSqr{\phi \to \varphi + \varphi + h} f_\phi(p_\phi) \qty[1 + f_\varphi(k_1)] \qty[1 + f_\varphi(k_2)],
    \end{aligned}
\end{equation}
where the factor of $1 / 2$ is the symmetry factor for two identical particles in the final state.

\paragraph{\boldmath Direct Integration with Dirac-$\delta$ Approximation}
With the Dirac-$\delta$ approximation for $f_\phi$ and $f_\varphi$ as
\begin{align}
    f_\phi(t, \rmp_\phi) & \simeq \frac{2 \pi^2}{\rmp_\phi^2} n_\phi(t) \delta(\rmp_\phi - \epsilon), \\
    f_\varphi\qty(t, \rmp_\varphi) & \simeq \frac{2 \pi^2}{\rmp_\varphi^2} n_\varphi(t) \delta\qty(\rmp_\varphi - \frac{m_\phi}{2} \frac{a_\ini}{a(t)}),
\end{align}
the collision term can be integrated directly as

\section{Numerical Results}

With $m_\phi = \tento{13} ~ \GeV$ and $\lambda = \tento{7} ~ \GeV$, we present the numerical results of $f_h(a, E_h)$ for $a / a_\ini = 10^6, 10^8, 10^{10}$ in Fig.~1.
In this case, one can see that the peak of $\tilde{f}_h\qty(a, \tilde{\rmk}_h)$ is around $\tilde{\rmk}_h \simeq m_\phi a_\ini / 2$, which may arise from the fact that $\tilde{f}_\varphi\qty(\alpha, \tilde{\rmp}_\varphi)$ peaks at $\tilde{\rmp}_\varphi \simeq m_\phi a_\ini / 2$ with an exponentially decreasing tail for $\tilde{\rmp}_\varphi > m_\phi a_\ini / 2$.
Therefore, we can consider to make an approximation to the numerical integration.

\begin{figure}
    \centering
    \includegraphics{code/plots/f_h_m1e13GeV_lambda1e7.pdf}
    \caption{$f_h(a, E_h)$ for $a / a_\ini = 10^6, 10^8, 10^{10}$ with $m_\phi = 10^{13} ~ \GeV$, $\lambda = 10^7 ~ \GeV$, respectively.}
\end{figure}

First, let us calculate the peak of 

\clearpage
\appendix\section*{Appendix}\addcontentsline{toc}{section}{Appendix}
\numberwithin{equation}{section}
\numberwithin{figure}{section}
\numberwithin{table}{section}

\section{\texorpdfstring{\TBD}{TBD.}}
\TBA

\printbibliography[heading=bibintoc]

\end{document}
