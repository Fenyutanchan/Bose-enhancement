% Copyright (c) 2025 Quan-feng WU <wuquanfeng@ihep.ac.cn>
% 
% This software is released under the MIT License.
% https://opensource.org/licenses/MIT

\documentclass{article}

\usepackage{amsmath}
% \usepackage{amsfonts}
\usepackage{amssymb}
\usepackage{authblk}
\usepackage[sorting=none, style=numeric-comp]{biblatex}
    \addbibresource{from_inspirehep.bib}
    \addbibresource{others.bib}
\usepackage[inline]{enumitem}
\usepackage{fontawesome5}
\usepackage[a4paper, margin=2cm]{geometry}
\usepackage{graphicx}
\usepackage[
    colorlinks=true,
    anchorcolor=violet,
    citecolor=orange,
    linkcolor=blue,
    urlcolor=magenta
]{hyperref}
% \usepackage[color={1 0 .5}]{attachfile2}
% \usepackage{mathrsfs}
\usepackage{mathtools}
\usepackage{physics}
\usepackage{slashed}
\usepackage{subcaption}
\usepackage{tensor}
\usepackage{xcolor}

\newcommand{\email}[1]{\footnote{E-mail: \href{mailto:#1}{#1}}}
\newcommand{\eg}{\textit{e.g.}}
\newcommand{\etc}{\textit{etc.}}
\newcommand{\ie}{\textit{i.e.}}
\newcommand{\license}{\footnote{This note © 2025 by \href{https://github.com/Fenyutanchan}{Quan-feng WU} is licensed under \href{https://creativecommons.org/licenses/by-nc-/4.0/}{CC BY-NC-ND 4.0}. \faCreativeCommons \faCreativeCommonsBy \faCreativeCommonsNc \faCreativeCommonsNd}}

\DeclareDocumentCommand{\TBA}{o}{{\color{red} TBA\IfNoValueTF{#1}{.}{: #1}}}
\DeclareDocumentCommand{\TBD}{o}{{\color{red} TBD\IfNoValueTF{#1}{.}{: #1}}}

\newcommand{\wqf}[1]{{\color{teal} [WQF: #1]}}
\newcommand{\notsure}[1]{{\color{green!60!black!100!} #1}}

\newcommand{\ee}{\mathrm{e}}
\newcommand{\ii}{\mathrm{i}}

\DeclareDocumentCommand{\twopi}{o}{\IfNoValueTF{#1}{2 \pi}{\qty(2 \pi)^{#1}}}
\DeclareDocumentCommand{\diracdelta}{om}{\delta\IfNoValueF{#1}{^{(#1)}}\qty(#2)}
\DeclareDocumentCommand{\twopidelta}{om}{\IfNoValueTF{#1}{\twopi[] \diracdelta{#2}}{\twopi[#1] \diracdelta[#1]{#2}}}
\DeclareDocumentCommand{\Heaviside}{sm}{\IfBooleanTF{#1}{\widetilde{\Theta}}{\Theta}\qty(#2)}

\newcommand{\GF}{G_\mathrm{F}}
\newcommand{\GN}{G_\mathrm{N}}
\newcommand{\MPl}{M_\mathrm{Pl}}
\newcommand{\kB}{k_\mathrm{B}}
\newcommand{\mPl}{m_\mathrm{Pl}}
\newcommand{\tPl}{t_\mathrm{Pl}}

\newcommand{\eV}{\mathrm{eV}}
\newcommand{\gram}{\mathrm{g}}
\newcommand{\meter}{\mathrm{m}}
\DeclareDocumentCommand{\second}{s}{\mathrm{s\IfBooleanT{#1}{ec}}}

\newcommand{\centi}{\mathrm{c}}
\newcommand{\kilo}{\mathrm{k}}
\newcommand{\mega}{\mathrm{M}}
\newcommand{\giga}{\mathrm{G}}
\newcommand{\tera}{\mathrm{T}}

\newcommand{\keV}{\kilo\eV}
\newcommand{\MeV}{\mega\eV}
\newcommand{\GeV}{\giga\eV}
\newcommand{\TeV}{\tera\eV}
\newcommand{\cm}{\centi\meter}
\newcommand{\kg}{\kilo\gram}

\newcommand{\tento}[1]{10^{#1}}
\newcommand{\timestento}[1]{\times \tento{#1}}

\newcommand{\hc}{\mathrm{h.c.}}

\newcommand{\tildedp}[1]{\widetilde{\dd{#1}}}

\newcommand{\BG}{\mathrm{BG}}
\newcommand{\BH}{\mathrm{BH}}
\newcommand{\PBH}{\mathrm{PBH}}
\newcommand{\QCD}{\mathrm{QCD}}
\newcommand{\tot}{\mathrm{tot}}

\newcommand{\element}[2]{\tensor[^{#2}]{\mathrm{#1}}{}}

\newcommand{\calC}{\mathcal{C}}
\newcommand{\calM}{\mathcal{M}}
\newcommand{\frakf}{\mathfrak{f}}
\newcommand{\rmk}{\mathrm{k}}
\newcommand{\rmp}{\mathrm{p}}
\newcommand{\rmq}{\mathrm{q}}

\newcommand{\amp}[1]{\calM_{#1}}
\newcommand{\ampNormSqr}[1]{\abs{\amp{#1}}^2}
\newcommand{\sumAmpNormSqr}[1]{\sum\ampNormSqr{#1}}
\newcommand{\avgAmpNormSqr}[1]{\overline{\ampNormSqr{#1}}}

\DeclareDocumentCommand{\CT}{oooo}{\calC\IfNoValueF{#2}{_{#2}}\IfNoValueF{#4}{^{#4}}\IfNoValueF{#1}{\qty[#1]}\IfNoValueF{#3}{\qty(#3)}}
\DeclareDocumentCommand{\Lagrangian}{oo}{\mathcal{L}\IfNoValueF{#1}{_{#1}}\IfNoValueF{#2}{\qty[#2]}}

\newcommand{\GW}{\mathrm{GW}}

\newcommand{\eff}{\mathrm{eff}}
\newcommand{\ini}{\mathrm{ini}}
\newcommand{\fin}{\mathrm{fin}}

\newcommand{\upini}{{(\ini)}}


\title{Quick Draft: Bose Enhancement in $\phi \to \varphi \varphi h$}
\author[a, b]{Quan-feng WU\email{wuquanfeng@ihep.ac.cn}}
\affil[a]{Institute of High Energy Physics, Chinese Academy of Sciences, Beijing 100049, CHINA}
\affil[b]{Kaiping Neutrino Research Center, Kaiping 529386, CHINA}
\date{\today\license}

\begin{document}

\maketitle

% \begin{abstract}
%     \TBA
% \end{abstract}
% \noindent\hrulefill

% \tableofcontents
\noindent\hrulefill
% \clearpage

\vspace{1em}

First, we have
\begin{equation}
    \sumAmpNormSqr{\phi \to \varphi \varphi h} = \frac{2 \lambda}{\MPl^2} \qty(1 - \frac{m_\phi}{2 E_h})^2.
\end{equation}
The corresponding collision term reads
\begin{align}
    \CT[f_\phi][\phi \to \varphi \varphi h][t, \rmp_\phi] = {} & \frac{1}{2 E_\phi} \int \tildedp{k_1} ~ \tildedp{k_2} ~ \tildedp{k_h} ~ \twopidelta[4]{p_\phi - k_1 - k_2 - p_h} \frac{\sumAmpNormSqr{\phi \to \varphi \varphi h}}{2} \nonumber \\
    & \qquad {} \times \qty{
        \begin{aligned}
            f_h(t, \rmk_h) ~ f_\varphi(t, \rmk_1) ~ f_\varphi(t, \rmk_2) ~ \qty[1 + f_\phi(t, \rmp_\phi)] \\
            {} - \qty[1 + f_h(t, \rmk_h)] ~ \qty[1 + f_\varphi(t, \rmk_1)] ~ \qty[1 + f_\varphi(t, \rmk_2)] ~ f_\phi(t, \rmp_\phi)
        \end{aligned}
    } \\
    \simeq {} & -\frac{f_\phi(t, \rmp_\phi)}{4 E_\phi} \int \tildedp{k_1} ~ \tildedp{k_2} ~ \tildedp{k_h} ~ \twopidelta[4]{p_\phi - k_1 - k_2 - p_h} \sumAmpNormSqr{\phi \to \varphi \varphi h} \nonumber \\
    & \qquad {} \times \qty[1 + 2 f_\varphi(t, \rmk_1) + f_\varphi(t, \rmk_1) ~ f_\varphi(t, \rmk_2)],
\end{align}
where in the last step we assumed that $f_h \ll 1$ for $\lambda^2 / \MPl^2$ suppressed graviton production.

After some lengthy calculations (including Dirac-$\delta$ function approximation for $f_\varphi$), we obtain the following results:
\begin{align}
    \CT[f_\phi][\phi \to \varphi \varphi h][t, \rmp_\phi][(1)] = {} & \frac{m_\phi f_\phi(t, \rmp_\phi)}{64 \twopi[3]} \frac{\lambda^2}{\MPl^2} \qty[2 \ln(2 \delta_h) + 3 - 8 \delta_h + 4 \delta_h^2] \tilde{\Theta}\qty(\delta_h \le \frac{1}{2}), \\
    \CT[f_\phi][\phi \to \varphi \varphi h][t, \rmp_\phi][(2)] = {} & -\frac{n_\varphi(t) f_\phi(t, \rmp_\phi)}{2 (\twopi) m_\phi^2} \frac{\lambda^2}{\MPl^2} \qty[\frac{a(t)}{a_\ini}]^2 \nonumber \\
    & \qquad {} \times \begin{dcases}
        \frac{a_\ini}{a(t)} \frac{1 - \frac{a_\ini}{2 a(t)}}{1 - \frac{a_\ini}{a(t)}} + \ln(1 - \frac{a_\ini}{a(t)}) & \delta_h \le \frac{1}{2} \qty[1 - \frac{a_\ini}{a(t)}], \\
        \frac{1}{4 \delta_h} + \ln(2 \delta_h) - \delta_h & \delta_h \ge \frac{1}{2} \qty[1 - \frac{a_\ini}{a(t)}],
    \end{dcases} \\
    \CT[f_\phi][\phi \to \varphi \varphi h][t, \rmp_\phi][(3)] = {} & -\frac{\pi n_\varphi^2(t) f_\phi(t, \rmp_\phi)}{4 m_\phi^5} \frac{\lambda^2}{\MPl^2} \qty[\frac{a(t)}{a_\ini}]^4 \qty[\frac{1 - \frac{2 a_\ini}{a(t)}}{1 - \frac{a_\ini}{a(t)}}]^2 \nonumber \\
    & \qquad {} \times \tilde{\Theta}\qty[\frac{a(t)}{a_\ini} \le 2] \tilde{\Theta}\qty(\delta_h \le 1 - \frac{a_\ini}{a(t)}),
\end{align}
where $0 < \delta_h \le 1/2$ is an infrared cutoff to regularize the soft graviton divergence.
Then, we apply that
\begin{equation}
    f_\phi(t, \rmp_\phi) = \frac{\twopi[2]}{2 \rmp_\phi} n_\phi(t) \delta\qty(\rmp_\phi - 0^+),
\end{equation}
which satisfies
\begin{equation}
    \int \frac{\dd[3]{\vb{p}_\phi}}{\twopi[3]} f_\phi(t, \rmp_\phi) = n_\phi(t).
\end{equation}
Therefore, the integrated collision terms read
\begin{align}
    \int \frac{\dd[3]{\vb{p}_\phi}}{\twopi[3]} \CT[f_\phi][\phi \to \varphi \varphi h][t, \rmp_\phi][(1)] = {} & \frac{m_\phi n_\phi(t)} {64 \twopi[3]} \frac{\lambda^2}{\MPl^2} \qty[2 \ln(2 \delta_h) + 3 - 8 \delta_h + 4 \delta_h^2] \tilde{\Theta}\qty(\delta_h \le \frac{1}{2}), \label{eq:integrated-collision-term-for-ϕ_TO_φφh-1} \\
    \int \frac{\dd[3]{\vb{p}_\phi}}{\twopi[3]} \CT[f_\phi][\phi \to \varphi \varphi h][t, \rmp_\phi][(2)] = {} & -\frac{n_\phi(t) n_\varphi(t)}{2 (\twopi) m_\phi^2} \frac{\lambda^2}{\MPl^2} \qty[\frac{a(t)}{a_\ini}]^2 \nonumber \\
    & \qquad {} \times \begin{dcases}
        \frac{a_\ini}{a(t)} \frac{1 - \frac{a_\ini}{2 a(t)}}{1 - \frac{a_\ini}{a(t)}} + \ln(1 - \frac{a_\ini}{a(t)}) & \delta_h \le \frac{1}{2} \qty[1 - \frac{a_\ini}{a(t)}], \\
        \frac{1}{4 \delta_h} + \ln(2 \delta_h) - \delta_h & \delta_h \ge \frac{1}{2} \qty[1 - \frac{a_\ini}{a(t)}],
    \end{dcases} \label{eq:integrated-collision-term-for-ϕ_TO_φφh-2} \\
    \int \frac{\dd[3]{\vb{p}_\phi}}{\twopi[3]} \CT[f_\phi][\phi \to \varphi \varphi h][t, \rmp_\phi][(3)] = {} & -\frac{\pi n_\phi(t) n_\varphi^2(t)}{4 m_\phi^5} \frac{\lambda^2}{\MPl^2} \qty[\frac{a(t)}{a_\ini}]^4 \qty[\frac{1 - \frac{2 a_\ini}{a(t)}}{1 - \frac{a_\ini}{a(t)}}]^2 \nonumber \\
    & \qquad {} \times \tilde{\Theta}\qty[\frac{a(t)}{a_\ini} \le 2] \tilde{\Theta}\qty(\delta_h \le 1 - \frac{a_\ini}{a(t)}). \label{eq:integrated-collision-term-for-ϕ_TO_φφh-3}
\end{align}

The Boltzmann transport equation reads
\begin{equation}
    \pdv{f_\phi}{t} - H \rmp_\phi \pdv{f_\phi}{\rmp_\phi} = \CT[f_\phi][][t, \rmp_\phi],
\end{equation}
whose integrated form is given by
\begin{equation}
    \dv{n_\phi}{t} + 3 H n_\phi(t) = \int \frac{\dd[3]{\vb{p}_\phi}}{\twopi[3]} \CT[f_\phi][][t, \rmp_\phi].
\end{equation}
With the change of variable $t \to a$, we have
\begin{equation}
    \dv{a}\qty(n_\phi a^3) = a H(a) \int \frac{\dd[3]{\vb{p}_\phi}}{\twopi[3]} \CT[f_\phi][][t, \rmp_\phi].
\end{equation}
Previously, we have considered only the $\phi \to \varphi \varphi$ decay channel, which gives the evolution of $n_\phi a^3$ as
\begin{equation}
    \dv{a}\qty(n_\phi a^3) = -\frac{\lambda^2}{32 \pi m_\phi a H(a)} \exp[\frac{\pi \lambda^2}{m_\phi^4 a^3 H(a)} \qty(n_\phi a^3)] \qty(n_\phi a^3).
\end{equation}
Also we have the decay width for $\phi \to \varphi \varphi$ as
\begin{equation}
    \Gamma_{\phi \to \varphi \varphi} = \frac{\lambda^2}{32 \pi m_\phi},
\end{equation}
which leads to
\begin{equation}
    \dv{a}\qty(n_\phi a^3) = -\frac{\Gamma_{\phi \to \varphi \varphi}}{a H(a)} \exp[\frac{\pi \lambda^2}{m_\phi^4 a^3 H(a)} \qty(n_\phi a^3)] \qty(n_\phi a^3).
\end{equation}
Therefore, far before $H \simeq \Gamma_{\phi \to \varphi \varphi}$, the comoving number density $n_\phi a^3$ is almost a constant, which can be evaluated as
\begin{equation}
    \eval{n_\phi a^3}_{H \gg \Gamma_{\phi \to \varphi \varphi}} = \min\qty{n_\phi^\upini a_\ini^3, \frac{m_\phi^4 H_\ini}{\pi \lambda^2} \ln(\frac{32 \pi m_\phi H_\ini}{\lambda^2})}.
\end{equation}
Now we take the case of
\begin{equation}
    \frac{m_\phi^4 H_\ini}{\pi \lambda^2} \ln(\frac{32 \pi m_\phi H_\ini}{\lambda^2}) \ll n_\phi^\upini a_\ini^3,
\end{equation}
which also means that the Bose enhancement effect is significant and thus the Universe is radiation-dominated by reheatons.
Now we can further consider the effect of $\phi \to \varphi \varphi h$ decay channel, which will further reduce $n_\phi a^3$.

Before proceeding, we should consider the initial Hubble parameter $H_\ini$.
The Universe backgound is depicted by the Friedmann-Lemaître-Robertson-Walker (FLRW) metric, which is a curved spacetime.
However, we are utilizing the flat-spacetime quantum field theory (QFT) in our calculations, which is only valid when the typical energy scale is much larger than the spacetime curvature scale \cite[p.~474]{Weinberg:2008zzc}, \ie, $H_\ini \ll m_\phi$ in our case.

Hereafter, $H$ is taken as the infrared cutoff for the same reason as above, which means $\delta_h \equiv H / m_\phi$ and $\delta_h^\upini \equiv H_\ini / m_\phi$.
Therefore, three major conditions should be considered regarding $\delta_h^\upini$:

\paragraph{\boldmath Case A: $\delta_h^\upini \le 1/4$.}
In this case, three stages can be identified according to Eqs.~\eqref{eq:integrated-collision-term-for-ϕ_TO_φφh-1}--\eqref{eq:integrated-collision-term-for-ϕ_TO_φφh-3}:
\begin{enumerate}[label=\alph*)]
    \item $1 \le a / a_\ini < \qty(1 - \delta_h^\upini)^{-1}$
    \item $\qty(1 - \delta_h^\upini)^{-1} \le a / a_\ini < \qty(1 - 2 \delta_h^\upini)^{-1}$
    \item $\qty(1 - 2 \delta_h^\upini)^{-1} \le a / a_\ini < 2$
\end{enumerate}

\paragraph{\boldmath Case B: $1/4 \le \delta_h^\upini < 1/2$.}
In this case, we have also three stages as:
\begin{enumerate}[label=\alph*)]
    \item $1 \le a / a_\ini < \qty(1 - \delta_h^\upini)^{-1}$
    \item $\qty(1 - \delta_h^\upini)^{-1} < a / a_\ini < 2$
    \item $2 \le a / a < \qty(1 - 2 \delta_h^\upini)^{-1}$
\end{enumerate}

\paragraph{\boldmath Case C: $1/2 \le \delta_h^\upini < 1$.}
Notice that the Hubble parameter $H$ will decrease as the Universe expands as
\begin{equation}
    H(a) = H_\ini \qty(\frac{a_\ini}{a})^2,
\end{equation}
which means that the Hubble parameter $H$ will eventually be smaller than $m_\phi / 2$, \ie, $\delta_h < 1/2$.
Form this point onward, the situation would be reduced to Case B with $H_\ini$ and $n_\phi^\upini$ replaced by $m_\phi / 2$ and the corresponding $n_\phi$ at that time, respectively.

% \clearpage
% \appendix\section*{Appendix}\addcontentsline{toc}{section}{Appendix}
% \numberwithin{equation}{section}
% \numberwithin{figure}{section}
% \numberwithin{table}{section}

% \section{\texorpdfstring{\TBD}{TBD.}}
% \TBA

\printbibliography[heading=bibintoc]

\end{document}
